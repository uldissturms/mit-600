
%
% 6.006 problem set 4
%
\documentclass[12pt,twoside]{article}

\input{macros}

\usepackage{amsmath}
\usepackage{url}
\usepackage{graphicx}
\usepackage{clrscode3e}

\newcommand{\answer}{
 \par\medskip
 \textbf{Answer:}
}

\newcommand{\collaborators}{ \textbf{Collaborators:}
%%% COLLABORATORS START %%%
Uldis Sturms & official solutions
%%% COLLABORATORS END %%%
}

\newcommand{\answerIa}{ \answer
%%% PROBLEM 1(a) ANSWER START %%%
3
%%% PROBLEM 1(a) ANSWER END %%%
}

\newcommand{\answerIb}{ \answer
%%% PROBLEM 1(b) ANSWER START %%%
4
%%% PROBLEM 1(b) ANSWER END %%%
}

\newcommand{\answerIc}{ \answer
%%% PROBLEM 1(c) ANSWER START %%%
7
%%% PROBLEM 1(c) ANSWER END %%%
}

\newcommand{\answerId}{ \answer
%%% PROBLEM 1(d) ANSWER START %%%
In the case of k < m O(N) amortized running time will not be preserved as table will have to be grown more than O(logN) times.
This way the cost of moving existing elements to a new array is not amortized. Doubling is a fast operation.
%%% PROBLEM 1(d) ANSWER END %%%
}

\newcommand{\answerIIa}{ \answer 
%%% PROBLEM 2(a) ANSWER START %%%
1
%%% PROBLEM 2(a) ANSWER END %%%
}

\newcommand{\answerIIb}{ \answer
%%% PROBLEM 2(b) ANSWER START %%%
3
%%% PROBLEM 2(b) ANSWER END %%%
}
